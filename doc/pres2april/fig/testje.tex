% By Fons van der Plas

\documentclass[]{article}\usepackage{tikz}\usepackage{pgfplots}\usepgfplotslibrary{fillbetween}\usepackage{amsmath,amssymb}\newcommand{\mat}[1]{\ensuremath{\boldsymbol{\mathrm{#1}}}}\newcommand{\mel}[1]{\ensuremath{{\mathrm{#1}}}}\newcommand{\phym}[1]{\ensuremath{\mathsf{#1}}}

\newcommand{\inlinebar}[2]{\begin{tikzpicture}
    \begin{axis}[
        grid=major,
        width=40mm,
        height=20mm,
        scale=1,
        xmin=0, xmax=1.2, ymin=0, ymax=1,
        xtick = {0,1}, ytick = {0.,1},
        xmajorticks=false,
        ymajorticks=false]
     \addplot[fill=#2, fill opacity=.3, draw=none, samples=500]
       plot (\x, {sign(#1-\x) } );
   \end{axis}
\end{tikzpicture}}


\newcommand{\inlineprob}[3]{\begin{tikzpicture}
    \begin{axis}[
        grid=major,
        width=40mm,
        height=20mm,
        scale=1,
        xmin=0, xmax=1.2, ymin=0, ymax=1,
        xtick = {0,1}, ytick = {0.,1},
        %xticklabel={\pgfmathparse{\tick*100}\pgfmathprintnumber{\pgfmathresult}\%},
        xmajorticks=false,
        ymajorticks=false]
     \addplot[fill=#3, fill opacity=.3, samples=500, draw=none]
       plot (\x, {1/(1+exp((\x-#1)*#2))} ) \closedcycle;
   \end{axis}
\end{tikzpicture}}

\newcommand{\Cross}{\mathbin{\tikz [x=1.4ex,y=1.4ex,line width=.2ex, red] \draw (0,0) -- (1,1) (0,1) -- (1,0);}}%

\definecolor{darkgreen}{rgb}{.1, .8, .1}
\newcommand{\Checkmark}{\color{darkgreen}\checkmark}

\begin{document}
Given a power injection $\mat{p}$, the Linearised Power Flow computes the saturation of each line.
This determines which lines will fail:
\[
\begin{pmatrix}
p_1\\
p_2\\
p_3\\
p_4
\end{pmatrix}
\quad
\tikz{\draw[|->,thick] (0,0) -- (20mm,0) node[above,midway] {LPF};}
\quad
\begin{pmatrix}
f_1 = \raisebox{-1mm}{\inlinebar{.5}{green}} \\
f_2 = \raisebox{-1mm}{\inlinebar{.2}{green}} \\
f_3 = \raisebox{-1mm}{\inlinebar{.8}{green}} \\
f_4 = \raisebox{-1mm}{\inlinebar{1.15}{red}} \\
f_5 = \raisebox{-1mm}{\inlinebar{.3}{green}}
\end{pmatrix}
\quad
\tikz{\draw[|->,thick] (0,0) -- (20mm,0) node[above,midway] {$\leq$ capacity};}
\quad
\begin{pmatrix}
\Checkmark\\
\Checkmark\\
\Checkmark\\
\Cross\\
\Checkmark
\end{pmatrix}
\]
When the power injection $\mat{P}$ is a stochastic variable, we get the saturation \emph{distribution} for each line. Large deviations theory estimates the \emph{failure probability}, and gives the \emph{most likely power injection $\hat{\mat{p}}$} that would cause that failure.
\[
\begin{pmatrix}
P_1\\
P_2\\
P_3\\
P_4
\end{pmatrix}
\quad
\tikz{\draw[|->,thick] (0,0) -- (20mm,0) node[above,midway] {LPF};}
\quad
\begin{pmatrix}
F_1 \sim \raisebox{-1mm}{\inlineprob{0.5}{10.0}{orange}} \\
F_2 \sim \raisebox{-1mm}{\inlineprob{0.25}{20.0}{orange}} \\
F_3 \sim \raisebox{-1mm}{\inlineprob{0.9}{100.0}{orange}} \\
F_4 \sim \raisebox{-1mm}{\inlineprob{1.1}{10.0}{orange}} \\
F_5 \sim \raisebox{-1mm}{\inlineprob{0.7}{5.0}{orange}}
\end{pmatrix}
\quad
\tikz{\draw[|->,thick] (0,0) -- (20mm,0) node[above,midway] {Large Dev.};}
\quad
\begin{pmatrix}
\phantom{0}0.001\%\phantom{0}\\
\phantom{0}0.0001\%\\
\phantom{0}0.1\%\phantom{000}\\
60.0\%\phantom{000}\\
10.0\%\phantom{000}
\end{pmatrix}, 
\begin{pmatrix}
\hat{\mat{p}}\,\mid \, F_1>1\\
\hat{\mat{p}}\,\mid \, F_2>1\\
\hat{\mat{p}}\,\mid \, F_3>1\\
\hat{\mat{p}}\,\mid \, F_4>1\\
\hat{\mat{p}}\,\mid \, F_5>1
\end{pmatrix}
\]

\end{document}