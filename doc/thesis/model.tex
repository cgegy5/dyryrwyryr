\documentclass[main.tex]{subfiles}

\begin{document}
In this chapter we will derive the \emph{Linear Power Flow} (LPF) transformation: a linear map from the vector of power injections at the buses to the vector of currents flowing through the lines of the network.

It is a \emph{right inverse} of $\FRT$, which means that applying $\LPF$ to a power injection gives a flow that would induce that injection. There are many possible candidates for a right inverse. For example, one way to construct one is to fix a minimum spanning tree; when we set line currents outside of the tree to zero, all line currents inside the tree are uniquely determined.

Yet, only one right inverse is \emph{physically correct}: the $\LPF$. This function can be derived explicitly by constructing an electric circuit that represents the whole transmission network and everything connected to it, and applying Ohm's Law and Kirchoff's Laws to relate line currents to power injections. We linearise the resulting \emph{node flow equations} to find the LPF matrix (page \pageref{eq:LPF}).

Using this linear map, we can transform the normal distribution of stochastic injections to a Gaussian distribution of line flows. Using the results of Chapter \ref{chap:probtheory}, we can estimate the overload probability of each line, resulting in a ranking of most vulnerable lines. Additionally, for each such line, we compute the most probable injection to cause the failure, and simulate the subsequent \emph{cascading failures}.

\section{The model}
The $\LPF$ is essentially the \emph{high-level model} used in the final chapters of this thesis (\ie the transmission network is modelled as a linear transformation). In this chapter, we derive a closed-form expression for the $\LPF$ from a lower-level, \emph{electrical} model. This derivation is based on the basic principles of \emph{AC circuit theory}, which extends the more familiar concepts of DC circuits (consisting of constant voltage and current sources and resistors) to circuits with time-varying (often sinusoidal) voltage and current sources. (This is where \emph{inductors} and \emph{capacitors} come into play.)

\emph{Electric Power Systems} by \cite{VonMeier2006} provides a very readable introduction into these subjects. Chapter 1 is an introduction to the physics of \emph{electricity}; Chapter 2 introduces \emph{DC circuit theory}; Chapter 3 concerns \emph{AC circuit theory}, specifically in the context of AC power transmission. 

For a fascinating, rigorous mathematical introduction into the subject, see \emph{Mathematical Foundations of Network Analysis} by \cite{Slepian1968} or \emph{Circuits, Matrices and Linear Vector Spaces} by \cite{huelsman1963circuits}.

\subsection{Electrical model}
We make a distinction between the \emph{structure} and the \emph{state} of the network. The structure is a directed graph, where each line is given an \emph{admittance}. The state collects the (real and reactive) power injected at each bus, the (complex) voltage of each bus, and the (complex) current flowing through each line.

The use of complex-valued voltage and current (and therefore power) is essential when analysing AC circuits, even though we are only interested in \emph{real} power. For example, we will find that the amount of real power transmitted over a line is approximately inversely proportional to the inductance of the line (the \emph{imaginary} equivalent of resistance), and approximately proportional to the difference in phase angles at its ends (the \emph{complex argument} of voltage).
% Feynman: What I cannot create, I do not understand.
%This approach attempts to combine mathematical and physical theory, while maintaining a clear distinction between the two.

\subsection{Time invariance}
The grid structure remains unchanged during normal operation, while the grid state is continually changing over time. For example, an important aspect of grid operation is \define{load profiling}: examining and predicting the total load connected to a node, as a function of time. As a result of changing loads, the flow of power in a grid is constantly changing.

One example of a change in grid structure is a \emph{line failure}, which can be modelled as the removal of an edge from the graph. In some cases, the removal of an edge from the graph results in an unconnected graph (\ie there exist two nodes with no sequence of lines connecting them). This scenario is called \emph{power islanding}\index{power island}.
Most transmission networks are designed in such a way that no single (or double) line failure can cause power islanding or a blackout, by increasing the \emph{edge-connectivity} of the graph.

In the case of a \emph{line overload}, however, a line failure is caused by an exceptionally high power flow, as a result of high supply or demand.\footnote{These high power injections are not necessarily located at the two endpoints of an overloaded line; it could also be a grid-wide pattern of power injections, all adding up to a high power flow on that line. We will study the \emph{most likely power injection} in Section \ref{sec:mostlikelypowerinjection}.}
The failure of an overloaded line will cause a redistribution of power flow, since the power flowing through the failed line now needs to 'find another path' between the nodes. In a highly stressed network, this redistribution can cause a second failure, which can then cause a third failure, and so on, which can eventually cause a blackout. We will study these \define{cascading failures} in Section \ref{sec:flowredistribution}.

\section{Grid structure}
A transmission grid is modelled as a directed graph $(\mathcal{N},\mathcal{L})$.%\todo{Require the graph to be connected? How about power islands?}
As vertices we take the \emph{nodes} of the network, which are those points where transmission lines connect to a generator, load, or to each other. Nodes are electrically distinct, in the sense that there is some non-zero impedance between them, allowing them to sustain a potential difference. In a network of $n$ nodes, they are represented by the natural numbers $1,\dots,n$, \ie $\mathcal{N}=\range{n}$.

A pair of two distinct nodes $(i,j)$ is contained in our set of lines $\mathcal{L}$ if there is a transmission line connecting the nodes. The choice of line orientation can be arbitrary, as long as we have $(i,j) \in \mathcal{L} \Rightarrow (j,i) \notin \mathcal{L}$ for all lines $(i,j) \in \mathcal{L}$. This transmission line has non-zero impedance. (Otherwise $i$ and $j$ would be the same node.) In a network of $m$ lines, lines are labelled $\mathcal{L}_1, \dots, \mathcal{L}_m$. 

In literature on the subject, buses\footnote{`bus' is short for `busbar': the latin `omnibus' in conjunction with `bar': some high-voltage cables are connected by welding them to a heavy metal bar.} are also called \emph{vertices} or \emph{nodes}, and lines can be called \emph{edges}, \emph{wires}, \emph{cables} or \emph{circuits}.

We model transmission lines as time-invariant impedances, which are assumed to be constant under any electric potential and current, allowing us to apply Ohm's Law. Instead of the impedance $\phym{Z}$ of a line, we use its \define{admittance} $\phym{Y=1/Z} \in \mathbb{C}$,\footnote{Written in Cartesian form, $\phym{Y=G}+i\phym{B}$, where $\phym{G}$ is the conductance, and $\phym{B}$ the susceptance of a line. Both have unit siemens (S), or mho (the reverse of 'ohm'), defined as $1 \,\si{\siemens}=1\,\si{\ohm}^{-1}$. }
 and Ohm's Law becomes:
\begin{align*}
    \mathsf{I=VY}
\end{align*}
This allows us to define the admittance between two unconnected nodes as $0$. (\ie no current is induced by a potential difference.) We define $\mat{\eta} \in \mathbb{C}^m$ as the \emph{admittance vector}\index{admittance!vector}, where $\eta_l$ is the admittance of $\mathcal{L}_l$, for each $l \in \range{m}$. 

\begin{definition}\label{def:gridstructure}
To summarise, an \emph{$(n,m)$-grid structure}\index{grid!structure} is defined as a tuple $((\mathcal{N},\mathcal{L}),\mat{K},\mat{\eta})$, where:
\begin{itemize}
    \item $n=\# \mathcal{N}$ is the number of nodes;
    \item $m=\# \mathcal{L}$ is the number of lines;
    \item $(\mathcal{N},\mathcal{L})$ is a connected directed graph with $\mathcal{N}=\range{n}$;
    \item $\mat{K} \in \mathbb{R}^{n \times m}$ is the vertex-edge incidence matrix of $(\mathcal{N},\mathcal{L})$;
    \item $\mat{\eta} \in \mathbb{C}^m$ is the line admittance vector.
\end{itemize}
\end{definition}

\section{Power grid state}
The \emph{state} of the network describes how the transmission network is being used (the (real and reactive) power injected at each node, and the voltage magnitudes) and how the electric circuit responds (voltage angles and (complex) line currents). More precisely, we use three physical quantities used in AC circuit analysis to describe the grid state:
\begin{definition}\label{def:gridstate}
A \emph{grid state}\index{grid!state} of an $(n,m)$-grid structure is defined as a tuple $(\mat{S}, \mat{V}, \mat{I})$, where
\begin{itemize}
    \item $\mat{S} \in \mathbb{C}^n$ is the \emph{complex power injection vector};
    \item $\mat{V} \in \mathbb{C}^n$ is the \emph{bus voltage vector};
    \item $\mat{I} \in \mathbb{C}^m$ is the \emph{line current vector}. (Currents are directed along digraph edges.)
\end{itemize}
\end{definition}

\section{State validity}
Given an $(n,m)$-grid structure $((\mathcal{N},\mathcal{L}),\mat{K},\mat{\eta})$, only some states are physically possible. A state that satisfies Kirchoff's Laws and Ohm's Law will be called \emph{valid}. Of course, since we are studying a real-world system, we are mainly interested in states that are valid, or at least close to being valid (in the sense of \emph{DC-valid}, see Section \ref{DCapproximation}).

\subsection{Circuit representation}
An $(n,m)$-grid structure $((\mathcal{N},\mathcal{L}),\mat{K},\mat{\eta})$ represents an electrical circuit, consisting of impedances and AC sources. A \emph{valid} grid state $(\mat{S}, \mat{V}, \mat{I})$ corresponds to a physical state of the circuit.
From the mathematical structure, we will construct the corresponding electric circuit in two layers, as follows:

In the first layer, each node $i$ in $\mathcal{N}$ becomes a node in the circuit. The value of $\mel{V}_i$ is the electric potential of that node, relative to ground.

A line $\mathcal{L}_k=(i,j)$ is modelled as an impedance element\footnote{complex-valued resistor} (\inlineres) with admittance $\mel{\eta}_k$ between the two nodes $i$ and $j$. The value of $\mel{I}_k$ is the current flowing through the impedance from $i$ to $j$.

See Figure \ref{fig:KVLcircuit} for the first constructed layer. This is not the final circuit: we have not yet added generators and loads to the circuit! Also, the transmission lines have no return wire. This means that there is no closed loop between two nodes, and therefore no energy can be transmitted.

To construct the second layer, we add a new component to each node $i \in \mathcal{N}$, which can be seen as the collection of AC sources (generators) and impedances (loads), connected in parallel between the node and ground. The exact way that loads and generators are connected to the node is not important,\footnote{This would be a circuit comprising the medium and low-voltage networks connected to the node, including every generator, fridge and phone charger that it serves. This is a common (and necessary) abstraction in power grid analysis.} so we will simply state that this component:
\begin{itemize}
    \item sustains a potential difference (which is $\mel{V}_i$);\footnote{In physical systems, the \emph{operating voltage} $|\mel{V}_i|$ (remember that $\mat{V}$ is complex) can be controlled by power plant operators by adjusting excitation current of a generator. The \emph{phase angle} $\theta_j=\Arg(V_j)$ can be controlled by adjusting the amount of energy (steam) supplied to a generator.

    Both methods are an indirect form of control,
    %and the effects are very complex in nature, since the operating voltage and phase angle of one node are inherently linked to that of 
	which also affects    
    all other nodes in the network. In fact, \emph{maintaining} a constant operating voltage and phase angle is a complicated task, requiring continuous adjustments to generator operation. Section 4.3 of \cite{VonMeier2006} covers this topic in more detail.}
    \item either supplies or draws a current, such that the amount of power generated or consumed by the component equals $\mel{S}_i$ (when $\mel{S}_i$ is positive or negative, respectively).
\end{itemize}
We will call this aggregation of generators and loads a \define{power injector} (\inlineac).

Each power injector is connected to a node on one end, and to a ground terminal on the other. In electric circuit theory, a ground terminal represents a direct connection to a universal ground: the `zero' reference of electric potential. One could say that between two different ground terminals, there exists a zero-impedance link connecting the two.
We now have a simple closed circuit between any two nodes connected by a transmission line, consisting of a power injector for the first node, an impedance between the two nodes, a power injector for the second node and a ground link back to the first node.

This zero-impedance `ground link' does not physically exist. Rather, it represents the ground wire of a three-phase transmission line. Because there is no return current and no electric potential across the return wire (see Section \ref{sec:threephase}), we can set its impedance to zero. Another interpretation is that the single high-voltage line \emph{represents the whole physical circuit} of three high-voltage wires and one return wire.

The complete, two-layer model is shown in Figure~\ref{fig:KVLcircuitside}.

\begin{figure}
    \centering
    \begin{adjustbox}{width=.6\textwidth}
    % By Fons van der Plas
% Based on "Interaction diagram" by Pascal Seppecher, which in turn is based on a diagram from Marco Miani.

%\documentclass[border=5mm]{standalone}\usepackage{tikz}\usepackage[european,siunitx]{circuitikz}\usepackage{amsmath,amssymb}\usetikzlibrary{positioning}\newcommand{\mat}[1]{\ensuremath{\boldsymbol{\mathrm{#1}}}}\newcommand{\mel}[1]{\ensuremath{{\mathrm{#1}}}}\newcommand{\phym}[1]{\ensuremath{\mathsf{#1}}}\begin{document}

\newcommand{\yslant}{0}
\newcommand{\xslant}{0}
\begin{circuitikz}[scale=1.1,every node/.style={minimum size=8mm},on grid]
\ctikzset{label/align = rotate}
	\draw[black, loosely dotted, thick] (0,.3) rectangle (10,7);
	
    % Top level
	\begin{scope}[
		yshift=0,
		every node/.append style={yslant=\yslant,xslant=\xslant},
		yslant=\yslant,xslant=\xslant
	]
	    \coordinate[](1hv) at (1,2);
	    \coordinate[](ihv) at (5,3);
	    \coordinate[](jhv) at (9,3);
	    \coordinate[](2hv) at (3.2,6);
	    
	    \coordinate[](xhv) at (10,3.7);
	    \coordinate[](yhv) at (10,2.2);
	    
	    \coordinate[](thv) at (0.5,.3);
	    \coordinate[](uhv) at (0,2.3);
	\end{scope}
    

	
	% HV level
	\begin{scope}[
		yshift=0,
		every node/.append style={yslant=\yslant,xslant=\xslant},
		yslant=\yslant,xslant=\xslant
	]
		
			
		\draw[] (1hv) to [R,l_=$\mel{Y}_{1,i}$, *-*] (ihv);
		\draw[thick] (jhv) to [R,l=$\mel{Y}_{i,j}$, semithick, *-*] (ihv);
		\draw[] (2hv) to [R,l_=$\mel{Y}_{2,i}$, *-*] (ihv);
		
		\draw[] (jhv) to [] (xhv);
		\draw[] (jhv) to [] (yhv);
		\draw[] (thv) to [] (1hv);
		\draw[] (uhv) to [] (1hv);
		
		\draw[fill=black]  
			(1hv) node[anchor=south] {$\mel{V}_1$}
			(ihv) node[anchor=south] {$\mel{V}_i$}
			(jhv) node[anchor=south] {$\mel{V}_j$}
			(2hv) node[anchor=south] {$\mel{V}_2$};
		
	\end{scope}
	
	
\end{circuitikz}
%\end{document}
    \end{adjustbox}
    \caption{A section of a transmission network, showing the line $(i,j)$. Two more nodes, both connected to $i$, are also shown. Note that this is only one layer of the electric circuit used in the model. In the full model (Figure \ref{fig:KVLcircuitside}), each transmission line forms a closed circuit, allowing current to flow.}
    \label{fig:KVLcircuit}
\end{figure}

%(\protect{\makebox[\width]{\raisebox{-1mm}{\begin{circuitikz}\ctikzset{bipoles/length=7mm}\draw[] (0,0) to[sV] (1,0);\end{circuitikz}}}})
\begin{figure}
    \centering
    \begin{adjustbox}{width=.9\textwidth}
    % By Fons van der Plas
% Based on "Interaction diagram" by Pascal Seppecher, which in turn is based on a diagram from Marco Miani.

%\documentclass[border=5mm]{standalone}\usepackage{tikz}\usepackage[european,siunitx]{circuitikz}\usepackage{amsmath,amssymb}\usetikzlibrary{positioning}\newcommand{\mat}[1]{\ensuremath{\boldsymbol{\mathrm{#1}}}}\newcommand{\mel}[1]{\ensuremath{{\mathrm{#1}}}}\newcommand{\phym}[1]{\ensuremath{\mathsf{#1}}}\begin{document}


\newcommand{\yslant}{.4}
\newcommand{\xslant}{-.7}
\begin{circuitikz}[scale=1.1,every node/.style={minimum size=8mm},on grid]
\ctikzset{label/align = rotate}

    % Ground level
	\begin{scope}[
		yshift=-120,
		every node/.append style={yslant=\yslant,xslant=\xslant},
		yslant=\yslant,xslant=\xslant
	]
	    \coordinate[](1gnd) at (1,2);
	    \coordinate[](ignd) at (5,3);
	    \coordinate[](jgnd) at (9,3);
	    \coordinate[](2gnd) at (3.2,6);
	    
	    \coordinate[](xgnd) at (10,3.7);
	    \coordinate[](ygnd) at (10,2.2);
	    
	    \coordinate[](tgnd) at (0.5,0.3);
	    \coordinate[](ugnd) at (0,2.3);
	\end{scope}
	
	
	
    % Top level
	\begin{scope}[
		yshift=0,
		every node/.append style={yslant=\yslant,xslant=\xslant},
		yslant=\yslant,xslant=\xslant
	]
	    \coordinate[](1hv) at (1,2);
	    \coordinate[](ihv) at (5,3);
	    \coordinate[](jhv) at (9,3);
	    \coordinate[](2hv) at (3.2,6);
	    
	    \coordinate[](xhv) at (10,3.7);
	    \coordinate[](yhv) at (10,2.2);
	    
	    \coordinate[](thv) at (0.5,.3);
	    \coordinate[](uhv) at (0,2.3);
	\end{scope}
    

	% Ground level
	\begin{scope}[
		yshift=-120,
		every node/.append style={yslant=\yslant,xslant=\xslant},
		yslant=\yslant,xslant=\xslant
	] 
	    \fill[white,fill opacity=.7] (0,.3) rectangle (10,7); % Opacity
		% The frame
		\draw[black, loosely dotted, thick] (0,0.3) rectangle (10,7);
		
		
		
		\draw[] (1gnd) to[short,*-*] (ignd) node[ground]{};
		\draw[thick] (jgnd) to[short,*-*] (ignd);
		\draw[] (2gnd) to[short,*-*] (ignd);
		
		\draw[] (jgnd) to [] (xgnd);
		\draw[] (jgnd) to [] (ygnd);
		\draw[] (tgnd) to [] (1gnd);
		\draw[] (ugnd) to [] (1gnd);
		
		 % Level title
		\fill[black]
			(0.2,6.5) node[right, scale=1] {\textbf{Ground}}	
			(5.1,1.9);
	\end{scope}
	
	% Vertical lines
	\draw[](1gnd) to[sV] (1hv);
	\draw[thick](ignd) to[sV, semithick] (ihv);
	\draw[thick](jgnd) to[sV, semithick] (jhv);
	\draw[](2gnd) to[sV] (2hv);
	
	
	\begin{scope}[
		yshift=0,
		every node/.append style={yslant=\yslant,xslant=0},
		yslant=\yslant,xslant=0
	]
	\draw[thick, blue, ->] (3.9,.8) arc (180:-170:1);
	\fill[black]
			(4.9,.8) node[blue, scale=1] {KVL}; 
	\end{scope}
	
	
	
	% HV level
	\begin{scope}[
		yshift=0,
		every node/.append style={yslant=\yslant,xslant=\xslant},
		yslant=\yslant,xslant=\xslant
	]
		% The frame:
		\fill[white,fill opacity=.7] (0,.3) rectangle (10,7); % Opacity
		\draw[black, loosely dotted, thick] (0,.3) rectangle (10,7);
		
			
		\draw[] (1hv) to [R,l_=$\mel{Y}_{1,i}$, *-*] (ihv);
	    \draw[thick] (jhv) to [R,l=$\mel{Y}_{i,j}$, semithick, *-*] (ihv);
		\draw[] (2hv) to [R,l_=$\mel{Y}_{2,i}$, *-*] (ihv);
		
		\draw[] (jhv) to [] (xhv);
		\draw[] (jhv) to [] (yhv);
		\draw[] (thv) to [] (1hv);
		\draw[] (uhv) to [] (1hv);
		
		\draw[fill=black]  
			(1hv) node[anchor=south] {$\mel{V}_1$}
			(ihv) node[anchor=south] {$\mel{V}_i$}
			(jhv) node[anchor=south] {$\mel{V}_j$}
			(2hv) node[anchor=south] {$\mel{V}_2$};
		 % Level title
		\fill[black]
			(0.2,6.5) node[right, scale=1] {\textbf{High-Voltage}}; 
	\end{scope}
	\draw[orange, rounded corners=3mm, dashed, thick] ($(ihv)-
	(1,1)$) rectangle ($(ihv) + (1,1)$);
	\node[orange,above] at ($(ihv) + (0,1)$){KCL};
	
	
\end{circuitikz}
%\end{document}
    \end{adjustbox}
    \caption{
    The electric model of the transmission network, showing the line $(i,j)$ and two other nodes. A node is represented by a single component (\inlineac), which is the aggregation of all generators and loads connected to that node. A transmission line is modelled as an impedance (\inlineres), a complex-valued resistor. The ground 'wires' have zero resistance. \protect\newline
    \textbf{KVL} is applied to each \textbf{line}, by traversing the loop drawn in blue.\protect\newline
    \textbf{KCL} is applied to each \textbf{node}, by summing all currents entering and leaving the orange area.}
    \label{fig:KVLcircuitside}
\end{figure}

\subsection{Kirchoff's Voltage Law (KVL) \& Ohm's Law}
For each line $\mathcal{L}_k=(i,j)$, we can apply Kirchoff's Voltage Law to the loop ``ground $\rightarrow$ $i$ $\rightarrow$ $j$ $\rightarrow$ ground'', as shown in Figure \ref{fig:KVLcircuitside}. This gives us:
$$\mel{V}_i + \Delta \mel{V}_k + -\mel{V}_j = 0,$$
where $\Delta \mel{V}_k$ is the electric potential of the line impedance. This potential relates to the line current according to Ohm's Law:
$$\mel{I}_k = \Delta \mel{V}_k \cdot \mel{\eta}_k = (\mel{V}_j - \mel{V}_i) \cdot \mel{\eta}_k.$$
Since the $k$th column of $\mat{K}$ (which corresponds to the line $\mathcal{L}_k=(i,j)$) has exacly two non-zero entries: $\mel{K}_{i,k}=1$ and $\mel{K}_{j,k}=-1$, we can write:
$$\mel{I}_k = (\mel{V}_j - \mel{V}_i) \cdot \mel{\eta}_k = -(\mel{K}^*{V})_{k}\mel{\eta}_k.$$
This holds for every $k \in \range{m}$, and this system of equations can be written compactly as:
$$\mat{I} = i\diag(i\mat{\eta})\mat{K}^*\mat{V}$$
Written in this form, we see that when $i\bm{\eta}, \mathbf{C}$ and $\mathbf{V}$ are purely real, then $\mathbf{I}$ is purely imaginary. Physically, this means that line current is always $90\si{\degree}$ out of phase with voltage differences. When phase angles are small, and voltage magnitude is constant, then the voltages `point roughly to the right' in the complex plane. Therefore, their \emph{differences} are almost purely imaginary.
\subsection{Kirchoff's Current Law (KCL)}
Using KVL and Ohm's Law, we found a relation between line current and node voltages. We can use KCL to relate the power injection to currents leaving and entering a bus.

We apply KCL to every high-voltage node in the electric circuit, as shown in Figure \ref{fig:KVLcircuitside}. For a bus $i \in \range{n}$, Kirchoff's Current Law states:
\[
\left[ \text{\textit{sum of currents leaving the node}}\right] \, - \, \left[\text{\textit{sum of currents entering the node}}\right] = 0.
\]
These currents are the currents of lines incident at the bus, together with the current `generated or consumed' by the power injector. Complex power is given by:
\[
\phym{S} = \conj{\,\phym{I}\,}\phym{V}.
\]
In our case, $\phym{I}$ is the current that we are looking for, flowing from the ground node to the high-voltage node at $i$, and $\phym{V}$ is the potential difference between the high-voltage node and ground, which is $\mel{V}_i$. The amount of power generated or consumed is given by the grid state: $\phym{S}=\mel{S}_i$. The current through the power injector is now given by $\conj{\mel{S}_i \mel{V}_i^{-1}}$.

The $i^{\text{th}}$ row of $\mat{K}$ corresponds to the bus $i$, and its entries are $1$ for lines leaving $i$, and $-1$ for lines entering $i$. This allows us to write the sum of currents in a compact way:
\[
\conj{\mel{S}_i \mel{V}_i^{-1}} + (\mel{K}\mel{I})_i = 0.
\]
Taking the complex conjugate and multiplying both sides by $\mel{V}_i$ gives $\mel{S}_i + \conj{(\mel{K}\mel{I})}_i \mel{V}_i = 0$ for each $i \in \range{n}$, or in matrix form:
$$\mat{S}+\conj{(\mat{K} \mat{I})}\pointwise \mat{V} = \mat{0}.$$
(The symbol $\pointwise$ denotes \emph{point-wise} multiplication.)
\subsection{Validity conditions}
Instead of \emph{requiring}\footnote{\cite{Slepian1968} gives an \emph{axiomatic} formulation of circuit theory.} the circuits laws to hold, we define them as an \emph{optional property} of the grid state.
\begin{definition}\label{def:statevalidity}
Given an $(n,m)$-grid structure $((\mathcal{N},\mathcal{L}),\mat{K},\mat{\eta})$, a grid state $(\mat{S}, \mat{V}, \mat{I})$ is \define{valid} if it satisfies the \define{KVL-Ohm equality}:
\begin{align}\label{eq:KVLOhmeq}
    \mat{I} = i\diag(i\mat{\eta})\mat{K}^*\mat{V} \tag{KVL-Ohm}
\end{align}
and the \define{S-KCL equality}:
\begin{align}\label{eq:KCLeq}
    \mat{S}+\conj{(\mat{K} \mat{I})}\pointwise \mat{V} = \mat{0}. \tag{S-KCL}
\end{align}
\end{definition}
\begin{remark}
In physical terms, the \emph{KVL-Ohm equality} states:
\begin{gather*}
    \textit{Each line $\mathcal{L}_k=(i,j)$ satisfies Ohm's Law ($\phym{I=VY}$), where:} \\
    \phym{I}=\mel{I}_k, \qquad \phym{V}=\mel{V}_{i} - \mel{V}_j, \qquad \phym{Y}=\mel{\eta}_k \nonumber
\end{gather*}
and the \emph{S-KCL equality} states:
\begin{gather*}
    \textit{At each node $i$, the sum of power injected at the node, $\mel{S}_i$}, \\
    \textit{and power injected from the grid, must equal $0$.} \nonumber
\end{gather*}
\end{remark}

\begin{proposition}
Given a grid structure and state as in Definition \ref{def:statevalidity}, the following are equivalent:
\begin{enumerate}[label=\roman*.]
    \item The grid state is valid.
    \item The grid state satisfies (\ref{eq:KVLOhmeq}) and (\ref{eq:KCLeq}).
    \item The grid state satisfies (\ref{eq:KVLOhmeq}), and each node $i$ satisfies the \define{node flow equation}, also called the \define{power mismatch equation}:
    \begin{empheq}[box=\fbox]{gather}
        \mel{S}_i = i\sum_{j=1}^{n} \conj{\mel{L}}_{i,j} |\mel{V}_i| |\mel{V}_{j}| e^{i(\theta_i - \theta_j)}\quad\quad\text{for each $i \in \range{n}$,}\label{eq:nodefloweq}\\
        \text{where }\mat{L}=\mat{K} \diag(i\mat{\eta}) \mat{K}^*
    \end{empheq}
    and $\mat{\theta} \in \mathbb{R}^n$ is the vector of \emph{voltage angles}, \ie $\theta_j=\Arg(V_j)$ (the principle argument of $\mel{V}_j$). We call $\mat{L}$ the \define{nodal susceptance matrix} \citep{Ronellenfitsch2017}.
\end{enumerate}

\end{proposition}
\begin{remark}
Note that (\ref{eq:nodefloweq}) does not depend on $\mat{I}$. This means that a valid state is uniquely determined by $\mat{S}$ and $\mat{V}$, since $\mat{I}$ can be computed from $\mat{V}$ using (\ref{eq:KVLOhmeq}).
\end{remark}


\begin{remark}
Writing in real and imaginary components, we get the node flow equations for real and reactive power for each node $i \in \range{n}$:
\begin{align}
    %\mel{S}_i &= \sum_{j=1}^{n} (\Re(\conj{\mel{M}}_{i,j})+i\Im(\conj{\mel{M}}_{i,j})) |\mel{V}_i| |\mel{V}_{j}| (\cos(\theta_i - \theta_j)+i\sin(\theta_i - \theta_j)) \\
    %\phym{P}=\Re (\mel{S}_i) &= \sum_{j=1}^{n} |\mel{V}_i| |\mel{V}_{j}| \Big[\Re(\mel{M}_{i,j})\cos(\theta_i - \theta_j)+\Im(\mel{M}_{i,j})\sin(\theta_i - \theta_j)\Big]
    \phym{P}=\Re (\mel{S}_i) &= \sum_{j=1}^{n} |\mel{V}_i| |\mel{V}_{j}| \Big[\Im(\mel{L}_{i,j})\cos(\theta_i - \theta_j)-\Re(\mel{L}_{i,j})\sin(\theta_i - \theta_j)\Big]
    \label{eq:realreactivenodefloweqreal}
    \\
    %\phym{Q}=\Im (\mel{S}_i) &= \sum_{j=1}^{n} |\mel{V}_i| |\mel{V}_{j}|\Big[\Re(\mel{M}_{i,j})\sin(\theta_i - \theta_j)-\Im(\mel{M}_{i,j})\cos(\theta_i - \theta_j)\Big]
    \phym{Q}=\Im (\mel{S}_i) &= \sum_{j=1}^{n} |\mel{V}_i| |\mel{V}_{j}|\Big[\Im(\mel{L}_{i,j})\sin(\theta_i - \theta_j)+\Re(\mel{L}_{i,j})\cos(\theta_i - \theta_j)\Big]
    \label{eq:realreactivenodefloweqreactive}
\end{align}
In literature on the subject, the node flow equation is often given in this form. The summands in the expression above are essentially the two-dimensional rotation matrix of angle $\theta_i - \theta_j$, applied to the vector $(\Re(i \conj{\mel{L}}_{i,j}), \Im(i \conj{\mel{L}}_{i,j}))^*$.

We will later study so called \emph{DC-valid} grid structures, where the values of $\mat{L}$ are real. If so, (\ref{eq:realreactivenodefloweqreal}) simplifies to:
\begin{align*}
    \phym{P}=\Re (\mel{S}_i) &= \sum_{j=1}^{n} \Re(\mel{L}_{i,j}) |\mel{V}_i| |\mel{V}_{j}| \sin(\theta_j - \theta_i).
\end{align*}
(Notice that we flipped $\theta_i$ and $\theta_j$.)
This tells us that in a network of just two nodes and one line with purely imaginary admittance, the amount of real power transmitted is proportional to $\sin (\theta_2-\theta_1)$.\footnote{The quantity $\theta_2 - \theta_1$ is called the \define{power angle} of the transmission line, a common measure of the amount of power being transmitted. A power angle greater than $45\si{\degree}$ will cause the nodes to lose synchronicity, making power transmission between the two nodes impossible. \citep{VonMeier2006} For long lines (over $100\si{\kilo\meter}$), this \define{stability limit} places an upper limit on the amount of power that a line can transmit. For shorter lines, the \emph{thermal limit} dominates.}


\end{remark}

\begin{proof}
(i) $\iff$ (ii) is true by definition.

Suppose the grid state satisfies (\ref{eq:KVLOhmeq}). We have
\begin{align*}
    \mat{S}+\conj{(\mat{K} \mat{I})}\pointwise \mat{V} &=\\
    \mat{S}+\conj{(i\mat{K} \diag(i\mat{\eta}) \mat{K}^* \mat{V})} \pointwise \mat{V} &= \\
    \mat{S}+\conj{(i\mat{L} \mat{V})} \pointwise \mat{V} &= \mat{0}
\end{align*}
iff for each $i\in\range{n}$
\begin{align*}
    \mel{S}_i &= -\conj{\left(\sum_{j=1}^{n} i\mel{L}_{i,j} \mel{V}_{j}\right)} \mel{V}_i \\
    &= -\left(\sum_{j=1}^{n} -i\conj{\mel{L}}_{i,j} \conj{\mel{V}}_{j}\right) \mel{V}_i \\
    &= i\sum_{j=1}^{n} \conj{\mel{L}}_{i,j} \mel{V}_i \conj{\mel{V}}_{j} \\
    &= i\sum_{j=1}^{n} \conj{\mel{L}}_{i,j} |\mel{V}_i| e^{i\theta_i} |\mel{V}_{j}| e^{-i\theta_j} \\
    &= i\sum_{j=1}^{n} \conj{\mel{L}}_{i,j} |\mel{V}_i| |\mel{V}_{j}| e^{i(\theta_i - \theta_j)},
\end{align*}
proving (ii) $\iff$ (iii).
\end{proof}

\section{Power Flow}
In the previous section, we derived a fundamental result: the \emph{node flow equation} (\ref{eq:nodefloweq}).
Recall from Section \ref{sec:powerflow} that the \emph{Power Flow problem} entails the following:
\begin{empheq}{gather*}
    \text{\emph{Given the production or consumption at each node,}}\\
    \text{\emph{\textbf{find the current flowing through each line.}}}
\end{empheq}
In the context of our electric model, this translates to:
\begin{empheq}{gather*}
    \text{\emph{Given an $(n,m)$-grid structure $((\mathcal{N},\mathcal{L}),\mat{K},\mat{\eta})$, and a power injection $\mat{S}$, }}\\
    \text{\emph{\textbf{find $\mat{V}$ and $\mat{I}$ such that $(\mat{S}, \mat{V}, \mat{I})$ is a valid state.}}}
\end{empheq}

It turns out that the easiest way to solve this problem is to solve the node flow equation, obtaining $\mat{V}$. Once $\mat{V}$ is known, one can easily compute $\mat{I}$ using \ref{eq:KVLOhmeq}, giving a state $(\mat{S}, \mat{V}, \mat{I})$ that is valid by construction.

The node flow equation is a system of non-linear equations, and no closed-form solution is known to exist in general. Fortunately, solving the node flow equation is essentially a root-finding problem. This means that well-established techniques, such as the Newton-Raphson algorithm, can be used to find a numerical solution.

The only difficulty lies in the \emph{number of unknowns} ($2n$ real numbers\footnote{Actually, the voltage angle $\theta$ of one bus (the \emph{slack bus}) is usually fixed to $0$, since \ref{eq:nodefloweq} only depends on \emph{differences} in voltage angles. We then have $2n-1$ unknowns.}) and finding an \emph{initial value} that converges to the solution. When studying the evolution of the grid state over time, we can use the solution of a previous iteration as initial value. In general, however, we need to come up with an initial guess.

The classical approach is to use the \define{flat start} as initial value: all voltage angles are set to $0$, and magnitudes are all set to the same value (say, $380 \, \si{\kilo\volt}$). For small networks, the Newton-Raphson algorithm then converges to a valid state, with small power angles between lines, and voltage magnitudes close to the initial value. For larger networks, however, this initial value rarely converges, and a valid is state can be `maddeningly difficult to obtain' \citep{Overbye2004}. 

Instead, a common approach is to approximate the grid structure and to linearises the laws of physics. The more approximations that we make, the easier it is to find a solution. A particular combination of approximations is known as the \emph{DC approximation}, in which case a \emph{closed-form} solution always exists, known as the \emph{Linear Power Flow}. For accurate power flow analysis, this solution is then used as initial value for the original system of equations. In this thesis, however, we will only use the solution of the Linear Power Flow, as it is easier to compute and analyse. This is common practice when studying \emph{cascading failures} \citep{Nesti2018emergentfailures, Ronellenfitsch2017, Purchala}.
\section{DC approximation}\label{DCapproximation}
`DC approximation' is a name given to a collection of assumptions/approximations, described below. The name `DC' refers to the approximation that the network is \emph{decoupled}. Unlike the abbreviation might suggest (DC usually stands for `Direct Current'), the power grid is still modelled as an AC (Alternating Current) network. The first version of this technique was published by \cite{Scott1974}, allowing the node flow equations to be solved efficiently using the computational power available at that time.\footnote{Their method does solve the actual node flow equation, but they optimised the iterative root-finding process by \emph{approximating the Jacobian}.}

Compare the following with Definitions \ref{def:gridstructure} and \ref{def:gridstate}.
\begin{definition}\label{def:DCaproximated}
Suppose $((\mathcal{N},\mathcal{L}),\mat{K},\mat{\eta})$ is an $(n,m)$-grid structure.
\begin{itemize}
    \item If $i\mat{\eta} \in \mathbb{R}^m \subseteq \mathbb{C}^m$  (\ie $\mat{\eta}$ has purely imaginary values) then the grid structure \emph{satisfies the DC approximation}\index{DC approximation!structure}.
    \footnote{\cite{Nesti2018emergentfailures} write $\beta=i\mat{\eta}$. Transmission line impedance ($\phym{Z=R+iX}$) is dominated by inductance, which is positive reactance  ($\phym{X}$). Resistance ($\phym{R}$) is always positive for passive components. Therefore, $\phym{Z}$ lies in the top-right quadrant of $\mathbb{C}$. Then the admittance, $\phym{Y=1/Z=G+iB}$, lies in the bottom-right quadrant of $\mathbb{C}$. In the DC approximation, line conductance ($\phym{G}$) is neglected, so ($\phym{Y=iB}$ with $\phym{B}<0$). Therefore, $\beta_k=i\eta_k=i\phym{Y}=-\phym{B}>0$ is the \emph{susceptance of line $k$, \textbf{with reversed sign.}}}
\end{itemize}
For a grid state $(\mat{S}, \mat{V}, \mat{I})$ on the structure:
\begin{itemize}
    \item If $\mat{S} \in \mathbb{R}^n \subseteq \mathbb{C}^n$ then the grid state \emph{satisfies the DC approximation}\index{DC approximation!state}. (Note that $\mat{V}$ and $\mat{I}$ need not be real-valued! We are still studying an AC circuit.)
    \item If $\mat{V} \in \mathbb{T}^n \cdot V_{op} \subseteq \mathbb{C}^n$ (\ie $|V_i|=V_{op}$ for each node $i$) for some \define{operating voltage} $V_{op} \in \mathbb{R}_{\geq 0}$, then the grid state admits a \define{flat profile}.

    In the special case $V_{op}=1$, the grid state admits a \define{normalised profile}.
\end{itemize}
\end{definition}
We note that $V_{op}=1$ does not necessarily mean that the transmission network is operating at $1 \, \si{\volt}$, it simply means that the network is operating at exactly \emph{one unit of electric potential} (which could be set to $1 \, \si{\volt}$, but also $380 \, \si{\kilo\volt}$, for example). This unit is known as a \define{power unit} (p.u.).

Compare the following with (\ref{eq:KVLOhmeq}) and (\ref{eq:KCLeq}).
\begin{definition}\label{def:approximatestatevalidity}
Given an $(n,m)$-grid structure $((\mathcal{N},\mathcal{L}),\mat{K},\mat{\eta})$ and a grid state $(\mat{S}, \mat{V}, \mat{I})$ that both satisfy the DC approximation. The grid state is \emph{approximately valid} if it satisfies the \emph{approximate KVL-Ohm equality}:
\begin{align}\label{eq:approxKVLOhmeq}
    \mat{I} = i\diag(i\mat{\eta})i\mat{K}^*\mat{\theta}V_{op} \tag{approx. KVL-Ohm}
\end{align}
and the \emph{approximate S-KCL equality}:
\begin{align}\label{eq:approxKCLeq}
    \mat{S}+\conj{(\mat{K} \mat{I})} V_{op} = \mat{0}. \tag{approx. S-KCL}
\end{align}
\end{definition}
The approximate S-KCL equality can be obtained by replacing the $\exp$ function in (\ref{eq:nodefloweq}) by $z \mapsto 1+z$ (the first two terms of the Maclaurin series of $\exp$):
\begin{proposition}\label{prop:approxnodeflow}
Given a grid structure state as in the previous definition, the state is approximately valid if and only if it satisfies the approximate KVL-Ohm equality and each node $i$ satisfies the \emph{approximate node flow equation}\index{node flow equation!approximate}:
\begin{empheq}[box=\fbox]{align}
    \mel{S}_i &= i\sum_{j=1}^{n} \conj{\mel{L}}_{i,j} |\mel{V}_i| |\mel{V}_{j}| (1+i(\theta_i - \theta_j))\quad\quad\text{for each $i \in \range{n}$.}\label{eq:approxnodefloweq}
\end{empheq}
\end{proposition}
\begin{proof}
Suppose the approximate KVL-Ohm equality hold. The approximate node flow equality holds if and only if for each node $i$, we have:
\begin{align*}
\mel{S}_i &= i\sum_{j=1}^{n} \conj{\mel{L}}_{i,j} |\mel{V}_i| |\mel{V}_{j}| (1+i(\theta_i - \theta_j)) \\
    &= i\sum_{j=1}^{n} \mel{L}_{i,j} |\mel{V}_i||\mel{V}_{j}| -  \sum_{j=1}^{n} \mel{L}_{i,j} |\mel{V}_i||\mel{V}_{j}|(\theta_i - \theta_j)\\
    \intertext{We assumed a DC state ($\mel{S}_i \in \mathbb{R}$):}
    &= -\sum_{j=1}^{n} \mel{L}_{i,j} |\mel{V}_i||\mel{V}_{j}|(\theta_i - \theta_j)\\
    &= -\sum_{j=1}^{n} \mel{L}_{i,j} |\mel{V}_i||\mel{V}_{j}|\theta_i + \sum_{j=1}^{n} \mel{L}_{i,j} |\mel{V}_i||\mel{V}_{j}|\theta_j\\
    &= -\theta_i |\mel{V}_i| \sum_{j=1}^{n} \mel{L}_{i,j}|\mel{V}_{j}| + |\mel{V}_i| \sum_{j=1}^{n} \mel{L}_{i,j} |\mel{V}_{j}|\theta_j\\
    \intertext{We assumed a flat profile ($|\mel{V}_i|=|\mel{V}_j|=V_{op}$ for every $i,j \in \range{n}$):}
    &= -\theta_i V_{op}^2 \sum_{j=1}^{n} \mel{L}_{i,j} + V_{op}^2 \sum_{j=1}^{n} \mel{L}_{i,j} \theta_j\\
	&= -\theta_i V_{op}^2 \sum_{j=1}^{n} \conj{\mel{L}}_{i,j} + V_{op}^2 \sum_{j=1}^{n} \mel{L}_{i,j} \theta_j\\
    \intertext{The rows of $\mat{L}$ add up to $0$:}
	&= 0 + V_{op}^2 \sum_{j=1}^{n} \mel{L}_{i,j} \theta_j\\
	&= V_{op}^2 \sum_{j=1}^{n} \mel{L}_{i,j} \theta_j\\
	&= (\conj{\mel{L}}\mel{\theta})_i V^2_{op} \\
	&= \conj{(\mel{K} \diag(i\mel{\eta})\mel{K}^*\mel{\theta})_i} V^2_{op} \\
	&= -\conj{(\mel{K} i\diag(i\mel{\eta})i\mel{K}^*\mel{\theta}V_{op})_i} V_{op} \\
	&= -\conj{(\mel{K} \mel{I})_i} V_{op}
\end{align*}
which is equal to $\mel{S}_i$ if and only if the approximate S-KCL equality holds.
\end{proof}

\begin{theorem}\label{thm:approxnodeflowlineq}
Suppose that a grid structure and state satisfy the DC approximation and that the state admits a flat, normalised profile. Then the approximated node flow equation is linear and real:
\begin{empheq}[box=\fbox]{align}
    \mat{S} &= \mat{L}\mat{\theta}.\label{eq:approxnodeflowlineq}
\end{empheq}
\end{theorem}

\begin{proof}
In the proof of Theorem \ref{prop:approxnodeflow}, we derived that the approximate node flow equation holds if and only if for each node $i$, we have:
\begin{align*}
    \mel{S}_i = (\conj{\mel{L}}\mel{\theta})_i V^2_{op}.
\end{align*}
Because $\mat{L}$ is real-valued and $V_{op}=1$, we find the result.
\end{proof}
%\towrite{Appendix: give a direct formula for $\mathbf{L}$}
%\towrite{compute $\mathbf{L}$ for the example networks}
%\towrite{give an alternative interpretation of $\mathbf{L}$ or $\mathbf{L}$: discrete laplacian}
%\subsection{Usefulness}
%Real transmission networks do not satisfy the DC approximation and, in general, a DC-valid state is not (physically) valid. Yet, using these two approximations has one important property: the approximated node flow equation becomes \emph{linear}. This has a number of advantages:
%\begin{itemize}
%    \item Solving the Power Flow problem (\ie computing line flows induced by a power injection vector) becomes trivial. As we will see in Section \ref{sec:LPFequations}, \towrite{ugh}
%    \item A linear power flow is crucial for the study of stochastic power injections, since it can be used to transform a probability distribution (like a multivariate normal distribution) without losing its structure.\todo{be specific}
%\end{itemize}

\subsection{Accuracy}
The DC approximation is a useful tool for understanding the complex nature of transmission networks. Therefore, it is crucial to verify that the DC approximation is, in fact, a good approximation, when real-world networks are studied. More precisely, one should ask:
\begin{enumerate}
    \item How close is a DC-valid state to being valid? More precisely, when the approximate node flow equation (\ref{eq:approxnodefloweq}) holds, what is the mismatch between the left hand and right hand side of the node flow equation (\ref{eq:nodefloweq})?
    \item How close are real-world grid structures to satisfying the DC approximation?
    %\item How does modifying a grid structure to satisfy the DC approximation affect the resulting power flow solution?
\end{enumerate}

The first question is answered in Proposition \ref{prop:powermismatchupperbound}, where an upper bound is derived for the power mismatch. It shows that the mismatch is bounded by the \emph{squares} of local differences in phase angles (i.e. $\theta_i - \theta_j$ for line $(i,j)$). \citet{Purchala} have shown that in the Belgian transmission network, all phase differences are below $7 \si{\degree}$, and $94\si{\percent}$ of lines have a phase difference below $2\si{\degree}$.

The second question is answered by \cite{Nesti2018emergentfailures}, who confirm that the SciGRID network (which we will study in Part II) satisfies the criteria found by \cite{Purchala}.

\begin{lemma}\label{lem:expaprrox}%\todo{Move to appendix?}
There exists $K \in \mathbb{R}_{\geq 0}$ such that
\begin{align*}
    |\exp ix - (1 + ix)| = K |x|^2 \qquad \text{for every $-2\pi \leq x \leq 2\pi$}.
\end{align*}
\end{lemma}%\todo{Find an upper bound for $K$, $K=0.5$ works}

\begin{proof}
We provide a proof using \emph{complex analysis}.\footnote{For an introduction, see \cite{GarlingVolIII}.} The complex (entire) function $z \mapsto \exp z$ is defined by the power series
\begin{align*}
    z\mapsto \sum_{k=0}^{\infty} \frac{z^k}{k!} = 1 + z + z^2\left(\frac{z^0}{2!}+\frac{z^1}{3!}+\dots\right)
\end{align*}%\todo{$0^0$}
which has infinite radius of convergence, and is continuous on $\mathbb{C}$.

The functions $z \mapsto \frac{1}{z^2}$, $z \mapsto \exp z$ and $z \mapsto 1+z$ are all holomorphic on $\mathbb{C}^*=\mathbb{C} \setminus \{0\}$, and so the function
\begin{align}
    g: \mathbb{C}^* \rightarrow \mathbb{C} \qquad \qquad
    g: z \mapsto \frac{1}{z^2}(\exp z - (1+z))
\end{align}
is holomorphic on $\mathbb{C}^*$, with a removable singularity at $0$. The (unique) extension of $g$ to $\mathbb{C}$ is an entire function, and by construction:
\begin{align*}
    \exp z = 1 + z + z^2g(z)\qquad \text{for each $z \in \mathbb{C}$}.
\end{align*}
Since $g$ is entire, it is continuous on $\mathbb{C}$.
$I=[-i2\pi, i2\pi]$ is a compact subset of $\mathbb{C}$, so $g$ is bounded on $I$, proving the result.
\end{proof}

\begin{proposition}\label{prop:powermismatchupperbound}
Suppose $((\mathcal{N},\mathcal{L}),\mat{K},\mat{\eta})$ is an $(n,m)$-grid structure, with a state $(\mat{S}, \mat{V}, \mat{I})$ that admits a flat profile with operating voltage $V_{op}$.

If the state is DC-valid, then there exists $K \in \mathbb{R}_{\geq 0}$ such that for each node $i$:
\begin{align*}
    \abs{\mel{S}_i - i\sum_{j=1}^{n} \conj{\mel{L}}_{i,j} |\mel{V}_i| |\mel{V}_{j}| e^{i(\theta_i - \theta_j)}}
    &\leq
    K V_{op}^2 \sum_{j=1}^{n} |\mel{L}_{i,j}| (\theta_i - \theta_j)^2 \\
    &\leq
    KV_{op}^2
    \norm{\mat{\eta}}_{\infty}  \norm{\mat{K}^*\mat{ \theta}}_2^2.
\end{align*}
%\todo{A more important result would be an upper limit for the error in \textit{line flow}, instead of the error in \textit{node flow}}
\end{proposition}
\begin{remark}
This result states that if the grid state is \emph{DC-valid}, and the power angles (i.e. $\theta_i - \theta_j$ for line $(i,j)$) are low, then the state is \emph{close} to also being \emph{valid}.

Quantitatively, it tells us that for a given node $i$, the 'error' resulting from using the approximate node flow equation (\ref{eq:approxnodefloweq}) is proportional to the \emph{squares} of phase angles of lines that connect to $i$.%\todo{this follows from the proof, not the statement itself}
\end{remark}

\begin{proof}
Suppose $i \in \range{n}$.

Choose a $K \in \mathbb{R}_{\geq 0}$ for which Lemma \ref{lem:expaprrox} holds.
The state is DC-valid, so we can substitute (\ref{eq:approxnodefloweq}) for $\mel{S}_i$:
\begin{align*}
    &\abs{\mel{S}_i - \sum_{j=1}^{n} \conj{\mel{M}}_{i,j} |\mel{V}_i| |\mel{V}_{j}| e^{i(\theta_i - \theta_j)}} \\
    &=
    \abs{
    i\sum_{j=1}^{n} \conj{\mel{L}}_{i,j} |\mel{V}_i| |\mel{V}_{j}| (1+i(\theta_i - \theta_j)) - i\sum_{j=1}^{n} \conj{\mel{L}}_{i,j} |\mel{V}_i| |\mel{V}_{j}| e^{i(\theta_i - \theta_j)}
    } \\
    &=
    V_{op}^2\abs{\sum_{j=1}^{n} \conj{\mel{L}}_{i,j}  \left(e^{i(\theta_i - \theta_j)} - (1+i(\theta_i - \theta_j))\right)} \\
    &\leq
    V_{op}^2\sum_{j=1}^{n} \abs{\mel{L}_{i,j}} \abs{e^{i(\theta_i - \theta_j)} - (1+i(\theta_i - \theta_j))} \qquad \text{(triangle inequality)}\\
    &\leq
    KV_{op}^2\sum_{j=1}^{n} \abs{\mel{L}_{i,j}} (\theta_i - \theta_j)^2  \qquad \text{(Lemma \ref{lem:expaprrox})}\\
    &=
    KV_{op}^2
    \left[
        \sum_{j=i} \abs{\mel{L}_{i,j}} (\theta_i - \theta_j)^2 +
        \sum_{\substack{j\neq i\\i,j \text{ connected}}} \abs{\mel{L}_{i,j}} (\theta_i - \theta_j)^2 +
        \sum_{\substack{j\neq i\\i,j \text{ not connected}}} \abs{\mel{L}_{i,j}} (\theta_i - \theta_j)^2
    \right]\\
    &=
    KV_{op}^2
    \left[
        \abs{\mel{L}_{i,i}} (\theta_i - \theta_i)^2 +
        \sum_{\substack{j\neq i\\i,j \text{ connected}}} \abs{\mel{L}_{i,j}} (\theta_i - \theta_j)^2 +
        \sum_{\substack{j\neq i\\i,j \text{ not connected}}} 0 \cdot (\theta_i - \theta_j)^2
    \right]\\
    &=
    KV_{op}^2
    \sum_{\substack{j\neq i\\i,j \text{ connected}}} \abs{\mel{L}_{i,j}} (\theta_i - \theta_j)^2
    \\
    &\leq
    KV_{op}^2
    \sum_{\mathcal{L}_k=(a,b) \in \mathcal{L}} \abs{\mel{L}_{a,b}} (\theta_a - \theta_b)^2
    \\
    &=
    KV_{op}^2
    \sum_{\mathcal{L}_k=(a,b) \in \mathcal{L}} \abs{\mel{\eta}_k} (\mel{K}^*\mel{\theta})_k^2
    \\
    &=
    KV_{op}^2
    \norm{\mat{\eta}  \pointwise \mat{K}^*\mat{\theta} \pointwise \mat{K}^*\mat{\theta}}_{1} \qquad \text{(in $\ell^1$)}
    \\
    &\leq
    KV_{op}^2
    \norm{\mat{\eta}}_{\infty}  \norm{\mat{K}^*\mat{\theta} \pointwise \mat{K}^*\mat{\theta}}_{1} \qquad \text{(Hölder's inequality)}
    \\
    &=
    KV_{op}^2
    \norm{\mat{\eta}}_{\infty}  \norm{\mat{K}^*\mat{\theta}}_2^2
\end{align*}
\end{proof}
%\towrite{Discuss the PDEs used for computing the PF?}
%\todo[inline]{Compare three methods for computing the PF in the two-node network:
%
%- original node flow equations
%
%- original node flow equations, but using the DC approximated Jacobian for root-finding
%
%- LPF
%
%they should be incrementally faster; the first two should find the same solution}
%
\section{Nodal susceptance matrix}
Theorem \ref{thm:approxnodeflowlineq} shows that the \emph{nodal susceptance matrix}, given by
\[
\mat{L} = \mat{K}\diag(i\mat{\eta})\mat{K}^*
\]
is not just useful for compact notation: it relates phase angles to power injection, taking the grid structure into account ($\mat{S} = \mat{L}\mat{\theta}$).

However, for our purposes, we are more interested in the \emph{inverse} of this function, which maps a power injection to the vector of phase angles that it induces at the buses. Unfortunately, the inverse does not exist: $\mat{L}$ is an $n \times n$ matrix, but it does not have full rank. This can be seen by the fact that $\mat{K}$ has rank $n-1$, so the rank of $\mat{L}$ is at most $n-1$. We can be more precise:

\begin{theorem}
Suppose that the entries of $i\mat{\eta}$ are non-zero.\footnote{An impedance element never has \emph{zero admittance}: this would be equivalent to having no element at all.} Then the nodal susceptance matrix, $\mat{L}$, has rank $n-1$, and its kernel has a one-element basis:
\[
\ker \mat{L} = \linspan \left\lbrace 
(1,1,\dots,1)^*.
\right\rbrace
\]
\end{theorem}
\begin{proof}
We have $\rank \mat{K} = n-1$ (Theorem \ref{thm:imageLPF}), so $\rank \mat{K}^* = \rank \mat{K} = n-1$. We will prove $\rank \mat{L} = n-1$ by showing that $\ker \mat{L} = \ker \mat{K}^*$. This tells us that they have the same \emph{nullity}, and by the Rank-Nullity Theorem, the result follows.

Suppose $\mat{\theta} \in \ker \mat{K}^*$. Then 
$$
\mat{L}\mat{\theta} = \mat{K}\diag(i\mat{\eta})\mat{K}^*\mat{\theta} = \mat{0},
$$
\ie $\mat{\theta} \in \ker \mat{L}$.

Conversely, suppose $\mat{\theta} \in \ker \mat{L}$. Then
\begin{align*}
\mat{L}\mat{\theta} &= \mat{0} \\
\Rightarrow \quad \mat{K}\diag(i\mat{\eta})\mat{K}^*\mat{\theta} &= \mat{0} \\
\Rightarrow \quad \mat{\theta}^* \mat{K}\diag(i\mat{\eta})\mat{K}^*\mat{\theta} &= 0 \\
\Rightarrow \quad (\mat{\theta}^* \mat{K}\diag(i\mat{\eta})\mat{K}^*\mat{\theta})^* &= 0 \\
\Rightarrow \quad (\diag(i\mat{\eta})^{\frac{1}{2}} \mat{K}^*\mat{\theta})^* (\diag(i\mat{\eta})^{\frac{1}{2}} \mat{K}^*\mat{\theta}) &= 0 \qquad \text{($\diag(i\mat{\eta})$ is self-adjoint)}\\
\Rightarrow \quad \norm{\diag(i\mat{\eta})^{\frac{1}{2}} \mat{K}^*\mat{\theta}} &= 0.
\end{align*}
All entries of $i\mat{\eta}$ are non-zero, so $\mat{K}^*\mat{\theta}$ must be zero. It follows that $\mat{\theta} \in \ker \mat{K}^*$.

The given basis has the right number of elements, and we can see from (\ref{eq:approxnodeflowlineq}) that $(1,1,\dots,1)^*$ is indeed an element of the kernel.
\end{proof}

\begin{corollary}\label{cor:imageL}
If the entries of $i\mat{\eta}$ are non-zero, then the \emph{image} of $\mat{L}$ is the set of all power injections with zero sum.
\end{corollary}
\begin{proof}
We have $\mat{L} = \mat{K}\diag(i\mat{\eta})\mat{K}^*$, so $\Ima \mat{L}$ is a linear subspace of $\Ima \mat{K}$. Applying Theorem~\ref{thm:imageLPF}, we find that $\Ima \mat{K}$ is the set of zero-sum injections, and that $\Ima \mat{K}$ has dimension $n-1$. The only linear subspace of $\Ima \mat{K}$ with dimension $n-1$ is $\Ima \mat{K}$ itself, so $\Ima \mat{L}=\Ima \mat{K}$, proving the result.
\end{proof}

In Theorem \ref{thm:approxnodeflowlineq}, we found the identity $\mat{S} = \mat{L}\mat{\theta}$, relating the power injection to the vector of voltage angles at the nodes. We now know that if $\mat{S}$ has zero sum, there is a set of phase angles that are mapped to $\mat{S}$ by $\mat{L}$. This set of solutions is a linear subspace of $\mathbb{R}^n$ of dimension $1$, and any two solutions differ by a constant angle.\footnote{This makes sense physically: increasing the phase angle at each node by the same amount makes no difference to the grid state, since power transmission is related to \emph{relative angles}: see Equation (\ref{eq:approxnodefloweq}).}

There are two paths to take after this point. One option is to fix the phase angle of the slack bus to $0$, which uniquely defines all other phase angles. This corresponds to taking the ${n-1 \times n-1}$ submatrix of $\mathbf{L}$, by removing the row and column of the slack bus. This submatrix is then inverted, and a row and column of all zeroes is added (corresponding to the slack bus). This method is used by \cite{PyPSA}.

A second method is to leave $\mat{L}$ as-is, and to consider the \emph{Moore-Penrose pseudoinverse} of $\mathbf{L}$. The resulting matrix, $\mat{L}^+$, has the property that \emph{if any solution to $\mat{S}=\mat{L}\mat{\theta}$ exists, one will be given by $\mat{L}^+\mat{S}$}.\footnote{In fact, the Moore-Penrose pseudoinverse solves the \emph{linear least squares} problem.} Corollary~\ref{cor:imageL} tells us that a solution exists if and only if $\mat{S}$ has zero sum (which is a realistic assumption). This is the approach of \cite{Nesti2018emergentfailures}, which they interpret as 'distributive slack': not fixing a single slack bus (we will discuss this concept in Section \ref{sec:nonzeroinjection}).

\section{Linear Power Flow}\label{sec:LPFequations}
Physically, there is a \emph{current} flowing along a line in the network, which has the function of transporting \emph{energy}. Although the current is alternating, the average effect is a \emph{flow of energy} from one bus to another. We define the \emph{power flow} along a line as the rate of flow of energy, which is constant when the power injection is constant.

In the context of line overloads, we are only interested in line \emph{current}, which can be seen as proportional to the amount of power being transmitted by the line. (Again, power is not a physical quantity that can be transmitted.) In a flat, normalised profile, power flow and line current are identical in magnitude. To follow general convention, we will talk about power flow instead of line current. For example, line ratings are often given in Watts, not Amperes.

Given an $(n,m)$-grid structure $((\mathcal{N},\mathcal{L}),\mat{K},\mat{\eta})$ and grid state $(\mat{S}, \mat{V}, \mat{I})$ that both satisfy the DC approximation, such that the state admits a flat profile, we define the \emph{power flowing through line $\mathcal{L}_k$} as:
\begin{align}
    \mel{f}_k = V_{op} (-i \mel{I}_k).\label{eq:powerflowdef}
\end{align}
This equation resembles\footnote{We multiply the current with $-i$ to account for the $90 \, \si{\degree}$ phase shift between current and voltage in a pure inductor.} the equation for electrical power ($\phym{P=VI}$), but it is not a quantity of power being generated or consumed in a component! Rather, it should be seen as a quantity symbolising the amount of power that a line transmits.\footnote{Alternatively, this definition can be seen as a \emph{change of unit} for electrical current.} Equation (\ref{eq:powerflowdef}) can be written in matrix form, giving the \emph{vector of line flows}:
\begin{align}
    \mat{f} = V_{op} (-i \mat{I}).
\end{align}

We assume a normalised profile (\ie $V_{op}=1$), and we substitute (\ref{eq:approxKVLOhmeq}) for $\mat{I}$: 
\begin{align*}
    \mat{f} = -iV_{op}\mat{I}=-iV_{op}^2i\diag(i\mat{\eta})\mat{K}^*\mat{\theta}
    = \diag(i\mat{\eta})\mat{K}^*\mat{L}^+\mat{p}.
\end{align*}
(We write $\mat{p}$ instead of $\mat{S}$ when $\mat{S}$ is real-valued.)

We find the \emph{Linear Power Flow equation}\index{Linear Power Flow}:
\begin{empheq}[box=\fbox]{gather}\label{eq:LPF}
    \mat{f}=\mat{F}\mat{p} \tag{LPF}\\[3mm]
    \text{where }\quad\mat{F}=\diag(i\mat{\eta})\mat{K}^*\mat{L^+} \notag
\end{empheq}

\emph{This is a linear transformation from a power injection vector $\mat{p}$ to the line flow $\mat{f}$ that induces it.}

If the line thresholds\index{line threshold} are $\mat{W}=(W_1, \dots, W_m)^* \in \mathbb{R}^m$, then we can define the \emph{normalised line flow}\index{line flow!normalised} as:
\begin{align*}
\hat{\mat{f}} = \diag(\mat{W})^{-1}\mat{f} = \hat{\mat{F}}\mat{p},
\end{align*}
where $\hat{\mat{F}} = \diag(\mat{W})^{-1}\mat{F}$ is the \emph{normalised LPF}.

\section{Stochastic power injections}\label{stochasticpowerinjections}
Using the linear transformation $\mat{F}$ that we just derived, we can compute the normalised line flow that a \emph{zero-sum} power injection $\mat{p}$ induces. This is a useful result: for example, one could check whether a generator configuration is \emph{admissible} by writing down the injection $\mat{p}$ associated to this configuration, and checking that no value of $\hat{\mat{F}}\mat{p}$ is greater in absolute value than $1$. (Since we \emph{normalised} the line flows, each line has unit threshold.) In a $(4,5)$-grid structure, this process might look something like:\footnote{Here, we are actually showing the \emph{absolute values} of $\hat{\mat{f}}$.}

\[
\begin{pmatrix}
\mel{p}_1\\
\mel{p}_2\\
\mel{p}_3\\
\mel{p}_4
\end{pmatrix}
\quad
\tikz{\draw[|->,thick] (0,0) -- (10mm,0) node[above,midway] {\begin{tabular}{c} \hphantom{.}norm. \\[-1mm] LPF \end{tabular}};}
\quad
\begin{pmatrix}
\hat{\mel{f}}_1 = \raisebox{-1mm}{\inlinebar{.5}{green}} \\
\hat{\mel{f}}_2 = \raisebox{-1mm}{\inlinebar{.2}{green}} \\
\hat{\mel{f}}_3 = \raisebox{-1mm}{\inlinebar{.8}{green}} \\
\hat{\mel{f}}_4 = \raisebox{-1mm}{\inlinebar{1.15}{red}} \\
\hat{\mel{f}}_5 = \raisebox{-1mm}{\inlinebar{.3}{green}}
\end{pmatrix}
\quad
\tikz{\draw[|->,thick] (0,0) -- (20mm,0) node[above,midway] {\begin{tabular}{c} below \\[-1mm] threshold? \end{tabular}};}
\quad
\begin{pmatrix}
\Checkmark\\
\Checkmark\\
\Checkmark\\
\TikzCross\\
\Checkmark
\end{pmatrix}
\]

Things get even more interesting when $\mat{p}$ is a \emph{stochastic variable}. In this case, $\hat{\mat{f}}$ is also a stochastic variable! In fact, given a probability distribution function for $\mat{p}$, we can use $\diag(\mat{W})^{-1}\mat{F}$ to compute the \emph{probability distribution function of $\hat{\mat{f}}$}. Using this probability distribution, we can now answer questions such as: ``What is the probability that line $i$ overloads?'' ($\PROB\left[|\hat{\mel{f}}_l| \geq 1 \right]$) Or: ``What is the probability that line $i$ overloads, given that line $j$ is operating at 95\% capacity?'' ($\PROB \left[ |\hat{\mel{f}}_i| \geq 1 \,\mid\, \hat{\mel{f}}_j = 0.95 \right]$). 

\[
\hspace{-7mm}
\begin{pmatrix}
\mel{p}_1\\
\mel{p}_2\\
\mel{p}_3\\
\mel{p}_4
\end{pmatrix}
%
\tikz{\draw[|->,thick] (0,0) -- (10mm,0) node[above,midway] {\begin{tabular}{c} \hphantom{.}norm. \\[-1mm] LPF \end{tabular}};}
%
\begin{pmatrix}
\hat{\mel{f}}_1 \sim \raisebox{-1mm}{\inlineprob{0.5}{10.0}{orange}} \\
\hat{\mel{f}}_2 \sim \raisebox{-1mm}{\inlineprob{0.25}{20.0}{orange}} \\
\hat{\mel{f}}_3 \sim \raisebox{-1mm}{\inlineprob{0.9}{100.0}{orange}} \\
\hat{\mel{f}}_4 \sim \raisebox{-1mm}{\inlineprob{1.1}{10.0}{orange}} \\
\hat{\mel{f}}_5 \sim \raisebox{-1mm}{\inlineprob{0.7}{5.0}{orange}}
\end{pmatrix}
\tikz{\draw[|->,thick] (0,0) -- (15mm,0) node[above,midway] {$\PROB \left[ \,\cdot\, \geq 1 \right]$};}
\begin{pmatrix}
\phantom{0}0.001\%\phantom{0}\\
\phantom{0}0.0001\%\\
\phantom{0}0.1\%\phantom{000}\\
60.0\%\phantom{000}\\
10.0\%\phantom{000}
\end{pmatrix}
\]

We will discuss the implications of having a stochastic power injection and the meaning of these absolute overload probabilities in Section \ref{sec:generationforecast}. For now, assume that we are studying the injection and flow during a time window long enough for line failures to occur,\footnote{\ie long enough for line protection mechanisms to have an effect. The classical assumption is that lines switch off within one AC cycle (\eg $20\,\si{\milli\second}$ for $50\,\si{\hertz}$) when overloaded.} but also short enough for fluctuations to be significant (and not `averaged out'). These assumptions should be seen as \emph{embodied in the bus covariance matrix}, but for the remainder of this chapter, we make no formal assumptions about the covariance matrix.

\subsection{Normally distributed power injection}
Following \cite{Nesti2018emergentfailures}, we model $\mat{p}$ to be \emph{multivariate normally distributed}:
\[
\mat{p} \, \sim \, \gaussdistr(\mat{\mu}_p, \mat{\Sigma}_{\mat{p}})
\]
where $\mat{\mu}_p$ is the mean power injection, which is the sum of deterministic generation and expected stochastic generation, minus the load. $\mat{\Sigma}_{\mat{p}}$ is the \emph{power injection correlation matrix}, which can be estimated from historical generation series. When power injection at different nodes is correlated (because of correlated weather), this matrix is non-diagonal.

Because $\hat{\mat{F}}$ is a linear transformation, the vector of normalised line flows $\hat{\mat{f}}$ is (multivariate) Gaussian distributed (Theorem \ref{thm:linearmapofgaussian}), and its distribution is given by

\begin{equation}
\hat{\mat{f}} \sim \gaussdistr(\mat{\mu_f}, \mat{\Sigma}_{\mat{f}}), \quad \text{ where } \quad
\mat{\mu_f} = \hat{\mat{F}}\mat{\mu_f} \quad \text{and} \quad
\mat{\Sigma}_{\mat{f}} = \hat{\mat{F}}\mat{\Sigma}_{\mat{p}}\hat{\mat{F}}^*.
\end{equation}

For realistic networks we have $m > n$, which means that $\hat{\mat{F}}$ is not surjective. Then $\mat{\Sigma}_{\mat{f}}$ is not injective, meaning that $\hat{\mat{f}}$ is Gaussian, but not normally distributed (Theorem \ref{thm:normaliffinvertible}).

\subsection{Overload probabilities}
For a line $l$, the probability distribution of the current through that line is simply the marginal distribution $\mel{f}_l$. Therefore, the probability of an emergent failure of line $l$ is given by: \footnote{Remember that the \emph{sign} of $\mel{f}_l$ corresponds to the \emph{direction} of current through the line. The line orientations were chosen arbitrarily, and only have meaning in our bookkeeping.}
\[
\PROB\left[|\mel{f}_l| \geq 1 \right] = \PROB\left[\mel{f}_l \leq -1 \right] + \PROB\left[\mel{f}_l \geq 1 \right]
\]
From Theorem \ref{prop:gaussianmarginaldistr} it follows that $\mel{f}_l \, \sim \, \gaussdistr(\mel{\mu}_{\mat{f}\,l}, \mel{\Sigma}_{\mat{f}\,ll})$, and $\PROB\left[|\mel{f}_l| \geq 1 \right]$ can now be computed using standard techniques.

The probability of \emph{any} emergent failure is $\PROB\left[\exists_{l \in \range{m}} |\mel{f}_l| \geq 1 \right]$, which can be approximated by an upper and lower bound:
\[
\max_{l \in \range{m}} \PROB\left[|\mel{f}_l| \geq 1 \right]
\quad\leq\quad
\PROB\left[\exists_{l \in \range{m}} |\mel{f}_l| \geq 1 \right]
\quad\leq\quad
\sum_{l \in \range{m}} \PROB\left[|\mel{f}_l| \geq 1 \right].
\]
Trivially, the most likely line flow, given that any emergent failure occurred, coincides with most likely line flow, given that the most vulnerable line failed.

\subsection{Most likely power injection}\label{sec:mostlikelypowerinjection}
Now that we have identified the most vulnerable lines in the network, we naturally want to simulate the effect that the failure of one of these lines will have. When we remove the line from our model, we get a new LPF matrix, which can be used to check the currents through all other lines, after the initial failure. But what power injection should be used? Because we are studying the \emph{hypothetical} failure of a line, we do not yet know the exact power injection that caused it.

The classical approach is to use the nominal power injection, $\mat{\mu}_{\mat{p}}$. This is exactly what we would do when studying regular line failures (caused by a fallen tree, for example). In our case, however, we assumed that the line failed because of an \emph{overload}, which tells us that the power injection must have deviated from its nominal value.

Because we have estimated a probability distribution for $\mat{p}$, we can find \emph{the most likely power injection, given that line $l$ overloaded}. We can compute this injection explicitly, leveraging the fact that the LPF map is linear.

\begin{theorem}\label{thm:mostlikelyinjection}
Suppose a grid with LPF $\hat{\mat{F}}$ has a $\gaussdistr(\mat{\mu}_p, \mat{\Sigma}_{\mat{p}})$-distributed power injection $\mat{p}$. The most likely power injection $\tilde{\mat{p}}^{(l)}$, given the emergent failure of a line $l$, is uniquely given by
\begin{empheq}[box=\fbox]{gather}\label{eq:mostlikelyinjection}
    \tilde{\mat{p}}^{(l)} = \mat{\mu}_{\mat{p}}  + \frac{\sign(\mel{\mu}_{\mat{f}\, l} ) - \mel{\mu}_{\mat{f}\, l} }{ \mel{\Sigma}_{\mat{f}\, ll}} \mat{\Sigma}_{\mat{p}} \hat{\mat{F}}^*\mat{e}_l
\end{empheq}
when $\mel{\mu}_{\mat{f}\, l} \neq 0$. Otherwise, there are two injections that maximise the conditional probability of $\mat{p}$, which are given by
\begin{align*}
    \tilde{\mat{p}}^{(l, +)} = \mat{\mu}_{\mat{p}}  + \frac{1}{ \mel{\Sigma}_{\mat{f}\, ll}} \mat{\Sigma}_{\mat{p}} \hat{\mat{F}}^*\mat{e}_l \quad \text{ and }\quad
    \tilde{\mat{p}}^{(l, -)} = \mat{\mu}_{\mat{p}}  + \frac{-1}{ \mel{\Sigma}_{\mat{f}\, ll}} \mat{\Sigma}_{\mat{p}} \hat{\mat{F}}^*\mat{e}_l.
\end{align*}
\end{theorem}
\begin{proof}

The set of power injections associated with the failure event of line $l$ is a union of two parallel \emph{planes} in $\mathbb{R}^n$. Indeed, the condition $\mel{f}_l = 1$ can be written as:
\[
\mel{f}_l = 1 \iff (\mel{F}\mel{p})_l = 1 \iff \mat{e}_l^*\hat{\mat{F}}\mat{p} = 1 \iff \left\langle \hat{\mat{F}}^*\mat{e}_l, \mat{p} \right\rangle = 1
\]
which is the equation defining the plane with pillar $\hat{\mat{F}}^*\mat{e}_l$. Similarly, the condition $\mel{f}_l = -1$ is satisfied if and only if $\mat{p}$ is contained in the plane with pillar $-\hat{\mat{F}}^*\mat{e}_l$.

We can now apply Theorem~\ref{thm:modeofaplaneconditional} to each pillar to find the mode of $\mat{p}$, given $\mel{f}_l = 1$, or $\mel{f}_l = 1$, respectively:
\begin{align}
\tilde{\mat{p}}^{(l, +)}
&=
\mat{\mu}_{\mat{p}}  + \frac{1 - \left\langle \mat{\mu}_{\mat{p}}, \hat{\mat{F}}^*\mat{e}_l \right\rangle}{\left\langle  \mat{\Sigma}_{\mat{p}} \hat{\mat{F}}^*\mat{e}_l, \hat{\mat{F}}^*\mat{e}_l \right\rangle} \mat{\Sigma}_{\mat{p}} \hat{\mat{F}}^*\mat{e}_l \notag \\
&=
\mat{\mu}_{\mat{p}}  + \frac{1 - \left\langle \hat{\mat{F}} \mat{\mu}_{\mat{p}}, \mat{e}_l \right\rangle}{\left\langle \hat{\mat{F}} \mat{\Sigma}_{\mat{p}} \hat{\mat{F}}^*\mat{e}_l, \mat{e}_l \right\rangle} \mat{\Sigma}_{\mat{p}} \hat{\mat{F}}^*\mat{e}_l \qquad \text{($\hat{\mat{F}}^*$ is the \emph{adjoint} of $\hat{\mat{F}}$)} \notag \\
&=
\mat{\mu}_{\mat{p}}  + \frac{1 - \left\langle \mat{\mu}_{\mat{f}}, \mat{e}_l \right\rangle}{\left\langle \mat{\Sigma}_{\mat{f}} \mat{e}_l, \mat{e}_l \right\rangle} \mat{\Sigma}_{\mat{p}} \hat{\mat{F}}^*\mat{e}_l \notag \\
&=
\mat{\mu}_{\mat{p}}  + \frac{1 - \mel{\mu}_{\mat{f}\, l} }{ \mel{\Sigma}_{\mat{f}\, ll}} \mat{\Sigma}_{\mat{p}} \hat{\mat{F}}^*\mat{e}_l \label{eq:modeinjectionpos}\\
\notag \\
\tilde{\mat{p}}^{(l, -)}
&=
\mat{\mu}_{\mat{p}} - \frac{1 - \left\langle \mat{\mu}_{\mat{p}}, -\hat{\mat{F}}^*\mat{e}_l \right\rangle}{\left\langle - \mat{\Sigma}_{\mat{p}} \hat{\mat{F}}^*\mat{e}_l, - \hat{\mat{F}}^*\mat{e}_l \right\rangle} \mat{\Sigma}_{\mat{p}} \hat{\mat{F}}^*\mat{e}_l \notag \\
&=
\mat{\mu}_{\mat{p}}  + \frac{-1 - \left\langle \mat{\mu}_{\mat{p}}, \hat{\mat{F}}^*\mat{e}_l \right\rangle}{\left\langle \mat{\Sigma}_{\mat{p}} \hat{\mat{F}}^*\mat{e}_l, \hat{\mat{F}}^*\mat{e}_l \right\rangle} \mat{\Sigma}_{\mat{p}} \hat{\mat{F}}^*\mat{e}_l \notag \\
& \;\, \vdots \notag \\
&=
\mat{\mu}_{\mat{p}}  + \frac{-1 - \mel{\mu}_{\mat{f}\, l} }{ \mel{\Sigma}_{\mat{f}\, ll}} \mat{\Sigma}_{\mat{p}} \hat{\mat{F}}^*\mat{e}_l. \label{eq:modeinjectionneg}
\end{align}
By symmetry of the marginal distribution of $\mel{f}_l$, it follows that in the unlikely case where $\mel{\mu}_{\mat{f}\, l}$ is zero, the line current is equally likely to deviate to the left as it is to deviate to the right, and both cases come with a different power injection. When $\mel{\mu}_{\mat{f}\, l}$ is non-zero, one of the two cases is more likely.
\begin{align*}
\mel{\mu}_{\mat{f}\, l} > 0 \, &\iff \, \tilde{\mat{p}}^{(l, +)}\text{ is the most probable injection,} \\
\mel{\mu}_{\mat{f}\, l} = 0 \, &\iff \, \tilde{\mat{p}}^{(l, +)} \text{ and }\tilde{\mat{p}}^{(l, +)}\text{ are the two most probable injections,} \\
\mel{\mu}_{\mat{f}\, l} < 0 \, &\iff \, \tilde{\mat{p}}^{(l, -)}\text{ is the most probable injection.}
\end{align*}
When $\mel{\mu}_{\mat{f}\, l} \neq 0$, we can use the $\sign$ function to combine Equations (\ref{eq:modeinjectionpos}) and (\ref{eq:modeinjectionneg}) into one, which gives the desired expression.
\end{proof}
%\towrite{Large dev: in het limiet $\epsilon \rightarrow 0$ valt de verwachtingswaarde van $\mathbf{p}$, gegeven emergent failure $l$, samen met de zojuist gegeven mode.}
%
%
\section{Redistribution of flow}\label{sec:flowredistribution}
In the previous section, we studied normally distributed power injections, and we can now discover which lines are likely to fail, and what power injection was the most likely cause. The next step is to study the \emph{redistribution of flow}: when overloaded lines are switched off (which happens almost instantly), the remaining lines in the network will have to take over their function, because \emph{the power injection remains unchanged after a line outage}.
In the simplest case of two parallel lines that connect two otherwise unconnected grids (say, a geographical island connected to the mainland via two cables), the failure of one line will force the other line to carry its original current, plus the current that would normally flow through the failed line.

Except for special cases like these, this \emph{redistributed flow} is, in general, hard to compute without the tools developed in this section. The general case not only consists of all possible line failures, we also want to study all \emph{possible combinations} of line failures.

We will discuss two methods to solve this problem: the Direct and the Optimised methods. The first method simply considers the graph obtained by removing the failed lines from the network, and then recalculates (\ref{eq:LPF}) for the network. Calculating the Moore-Penrose inverse of $\mat{L}$ is computationally expensive\footnote{Scientific computing libraries generally calculate $\mat{L}^+$ using the Singular Value Decomposition of $\mat{L}$.}, and every combination of line failures requires this calculation.\footnote{Calculating the LPF of the SciGrid network ($n=489$, $m=895$) takes approximately $700 \, \si{\milli\second}$, excluding the additional overhead of copying the unperturbed network.} A second method, first introduced by \cite{Guler2007}, utilises the LPF of the original network to derive the redistribution of flow. This Optimised method is computationally less expensive, and provides additional insight into the effect of line outages, that would not be obtained when discarding the original network.

\subsection{Direct method}
For the Direct method, we simply recompute the LPF for the perturbed network. To avoid reducing the dimension $m$, and recalculating $\mat{K}$, we set the admittance of each failed line to zero.

More formally, suppose $((\mathcal{N},\mathcal{L}),\mat{K},\mat{\eta})$ is an $(n,m)$-grid structure with $\mathcal{Z} \subseteq \range{m}$ a collection of $v \in \range{m}$ lines that fail. For an injection $\mat{p} \in \mathbb{R}^n$, the line flows \emph{before} the failures of $\mathcal{Z}$ are given by:
\begin{align}
\mat{f}^{\mathcal{Z}} &= \diag(i\mat{\eta})\mat{K}^*\left(\mat{K}\diag(i\mat{\eta})  \mat{K}^*\right)^+\mat{p}
\intertext{(by definition of $\mat{F}$) and the line flows \emph{after} the failures are given by:}
\mat{f}^{\mathcal{Z}} &= \diag(i\mat{\eta})\mat{K}^*\left(\mat{K}\diag(i\mat{\eta})\mat{I}_{\range{m} \setminus \mathcal{Z}}\mat{K}^*\right)^+\mat{p},
\end{align}
where $\mat{I}_{\range{m} \setminus \mathcal{Z}}$ is the identity matrix, with diagonal entries set to zero for line numbers contained in $\mathcal{Z}$. By taking the product $\diag(i\mat{\eta})\mat{I}_{\range{m} \setminus \mathcal{Z}}$, we are essentially setting the admittance of failed lines to zero, which corresponds physically to a circuit break.
\subsection{Optimised method}\label{sec:optimisedmethod}
\begin{theorem}\label{thm:lineflowsafterfailures}
Suppose $((\mathcal{N},\mathcal{L}),\mat{K},\mat{\eta})$ is an $(n,m)$-grid structure, with LPF $\mat{F}$. Suppose that $\mathcal{Z} \subseteq \range{m}$ is a collection of $v \in \range{m}$ lines that fail. For a given injection $\mat{p} \in \mathbb{R}^n$, the line flows \emph{before} the failures of $\mathcal{Z}$ are given by:
\begin{align}
\mat{f} &= \mat{F} \mat{p} \\
\end{align}
and the line flows \emph{after} the failures are given by:
\begin{empheq}[box=\fbox]{gather}
\mat{f}^{\mathcal{Z}} = \mat{f} - \mat{M}\mat{N}\left(\mat{N}^*\mat{M}\mat{N}\right)^+\mat{N}^*\mat{f}\label{eq:lineflowsafterfailures}
\end{empheq}
where $\mat{M} = \mat{F}\mat{K} - \mat{I} \in \mathbb{R}^{m \times m}$ and $\mat{N}=\left(\mat{e}_{\mathcal{Z}_1} \cdots \mat{e}_{\mathcal{Z}_v}\right) \in \mathbb{R}^{m \times v}$, the matrix that is zero everywhere, except for the entries $\left(\mathcal{Z}_i, i\right)$ (for $i \in \range{v}$), where it has value $1$.
\end{theorem}
\begin{remark}
This expression for $\mat{f}^{\mathcal{Z}}$ \emph{also} contains a pseudo-inverse, which might not look computationally advantageous, compared to the Direct method. Note, however, that right multiplying by $\mat{N}$ corresponds to taking the \emph{submatrix of column numbers} $\mathcal{Z}_1$ through $\mathcal{Z}_v$, and left multiplying by $\mat{N}^*$ gives the submatrix of \emph{row} numbers in $\mathcal{Z}$. This means that $\mat{M}$ only needs to be computed once, after which most matrix multiplications in (\ref{eq:lineflowsafterfailures}) can done by \emph{indexing appropriately}, for any combination of line failures.
Additionally, if the number of failed lines is small, the product $\mat{N}^*\mat{M}\mat{N}$ will be a small matrix, drastically improving performance.
Lastly, this expression only requires (pseudo-)solving the system of equations $\left(\mat{N}^*\mat{M}\mat{N}\right)\mat{\alpha}=\mat{N}^*\mat{f}$, which can be done more efficiently than computing the full pseudo-inverse. (The same optimisation can be also applied to the Direct method.)
\end{remark}

Several proofs to this theorem exist. The first proof, by \cite{Guler2007}, iterates over the failed lines, proving the result by natural induction. \citep{Guo2009} provide two additional proofs, which follow from a careful analysis of the Direct method, where they \emph{remove the columns of $\mat{K}$} that correspond to failed lines. An entirely different, fourth proof, due to \cite{Ronellenfitsch2017}, uses the formalism of \emph{graph cycles}, which form a basis for $\ker \mat{K}$. Their article is truly fascinating, as it explores how flow redistributions are composed of loop flows. Additionally, they derive a more general result, where the grid perturbation is expressed as a \emph{change in line admittances $\mat{\eta}$}.\footnote{Some transmission networks deploy adjustable inductors that clamp onto transmission lines, to \emph{steer} the flow of current, making this generalisation especially relevant.} Taking the limit $\mel{\eta}_l \rightarrow 0$ then corresponds to the removal of the line.

Before providing this fourth proof, we will try to deduce the result from intuitive reasoning, to better understand the found expression.

\begin{intuition}
Both $\mat{f}$ and $\mat{f}^{\mathcal{Z}}$ are induced by the same power injection, \ie $\mat{K} \mat{f} = \mat{K} \mat{f}^{\mathcal{Z}} = \mat{p}$. This means that the difference between the two, which is the change in flow right after the line failures, is an \emph{element of the kernel of $\mat{K}$}. We denote this difference by
$\Delta\mat{f} \defeq \mat{f}^{\mathcal{Z}} - \mat{f}$.

For each $l \in \mathcal{Z}$, the current through line $l$ must become zero. This imposes the condition
\begin{align}
\Delta\mel{f}_l = \mel{f}^{\mathcal{Z}}_l - \mel{f}_l = -\mel{f}_l,\label{eq:diffcondition}
\intertext{or more compactly,}
\mat{N}^*\Delta\mat{f} = -\mat{N}^*\mat{f}.\label{eq:diffconditionmat}
\end{align}
Let us first consider the case of a single failure: $\mathcal{Z}=\left\{l\right\}$, that does not cause the network to become disconnected. We are looking for a $\Delta\mat{f} \in \ker \mat{K}$, under the condition that $\mel{f}^{\mathcal{Z}}_l = -\mel{f}_l$. The first choice that might come to mind is to fix a basis of $\ker \mat{K}$ consisting of \emph{unit loop flows} in the graph. We can pick a unit loop flow that is non-zero at $l$, and scale by $\pm \mel{f}_l$ to find the desired flow difference.

Although this does satisfy the condition at $l$, it is in general not the flow that will be \emph{induced} in the perturbed network, which is uniquely determined by power flow physics. There are many linear combinations of unit loop flows that satisfy the condition at $l$, one of which is the correct one.

Suppose that the network is unused ($\mat{p}=\mat{0}$). We now \emph{force} a current of $1$ through line $l$ and we fix all other line currents to $0$. This line flow vector is given by $\mat{e}_l$. The result of this flow will be a power injection which remains zero everywhere, except at the two nodes that $\mathcal{L}_l = (i,j)$ connects. This power injection is given by $\mat{K}\mat{e}_l$.

On the other hand, if we were to apply the injection $\mat{K}\mat{e}_l$ to the network, allowing current to flow naturally, we would find the vector of line currents $\mat{F}\mat{K}\mat{e}_l$, which in general does not equal $\mat{e}_l$! Of course, the natural line flow in $l$ will still be relatively large, but some power will be transmitted along different routes. For example, in a circular network of four buses and four lines of equal admittance, with $l=1$, we find\footnote{The series combination of lines 2, 3 and 4 has a third of the admittance of line 1, so their current must equal a third of the current through line 1.}
\begin{align*}
\mat{F}\mat{K}\mat{e}_1 = \begin{pmatrix}
\hphantom{-}0.75 \\
-0.25 \\
-0.25 \\
-0.25
\end{pmatrix},
\text{ with difference }
\mat{F}\mat{K}\mat{e}_1 - \mat{e}_1 = \begin{pmatrix}
-0.25 \\
-0.25 \\
-0.25 \\
-0.25
\end{pmatrix}.
\end{align*}
Because $\mat{F}$ is the right-inverse of $\mat{K}$, we find that $\mat{K} \left(\mat{F}\mat{K}\mat{e}_l\right) = \mat{K}\mat{e}_l$, which means that the \emph{difference between a natural flow and a forced flow}, $\mat{F}\mat{K}\mat{e}_1 - \mat{e}_1$, is an element of the kernel of $\mat{K}$.

We have not yet satisfied the condition (\ref{eq:diffconditionmat}). Given a power injection $\mat{p}$ and unperturbed flow $\mat{f}=\mat{F}\mat{p}$, we can \emph{scale} the above difference with some $\alpha \in \mathbb{R}$:
\begin{align*}
\Delta\mat{f} = \left(\mat{F}\mat{K}\mat{e}_l - \mat{e}_l\right)\alpha
\intertext{with $\alpha$ such that}
\Delta\mel{f}_l = -\mel{f}_l
\end{align*}
is satisfied. If the network remains connected, $\alpha$ is given by $-\mel{f}_l/\left(\mel{F}\mel{K}\mel{e}_l - \mel{e}_1\right)_l = \mel{f}_l/\left(1 - (\mel{F}\mel{K})_{ll}\right)$.
%\todo{Het voorbeeld van een circulair netwerk is te simpel}
We state, \emph{without proof}, that this is indeed the flow difference dictated by power flow physics.\footnote{If we assume the Theorem to be true, one could work backwards from (\ref{eq:lineflowsafterfailures}) to find this result.}

In our above example, we find $\alpha = 3$, which gives
\begin{align*}
\Delta\mat{f} = \begin{pmatrix}
-0.75 \\
-0.75 \\
-0.75 \\
-0.75
\end{pmatrix},
\text{ and the \emph{redistributed flow}: }
\mat{f}^{\mathcal{Z}} = \mat{f} + \Delta\mat{f} = \begin{pmatrix}
\hphantom{-}0.00 \\
-1.00 \\
-1.00 \\
-1.00
\end{pmatrix}.
\end{align*}
Of course, we could have found this result quite easily: after the first line fails, the unit of power simply traverses the loop in the other direction (hence the flipped sign in $\mat{f}^{\mathcal{Z}}$).

In the general case of multiple line failures, we essentially apply the above procedure to each failed line, and add the resulting differences. If we were to use this method \emph{iteratively}, by considering the failed lines in some chosen order, we run into the following issue, when more than one line is removed: when removing the second line from the network, we will find a difference flow that sets the second line current to zero. Unfortunately, this difference flow will also change the current of the first line, which is then no longer zero. If we then consider the first line again, we will also affect the second line, et cetera.

To avoid this cat-and-mouse game, we need to find all scaling factors $\alpha_1, \dots, \alpha_v$ \emph{simultaneously}. By virtue of linearity, the $v$ conditions imposed by (\ref{eq:diffconditionmat}) form a set of \emph{linear conditions} on $\mat{\alpha} = (\alpha_1, \dots, \alpha_v)^*$.

For each line $l \in \mathcal{Z}$, the natural-forced difference is $\Delta\mat{f}^l \defeq \mat{F}\mat{K}\mat{e}_l - \mat{e}_l$. When combining all differences for $l \in \range{m}$ as columns of a matrix, we find:
\begin{align*}
\begin{pmatrix}
\mid & & \mid \\
\Delta\mat{f}^1 & \cdots & \Delta\mat{f}^m\\
\mid & & \mid
\end{pmatrix} = \mat{F}\mat{K} - \mat{I} = \mat{M}
\end{align*}
The sub-matrix of differences for $l \in \mathcal{Z} \subseteq \range{m}$ is given by:
\begin{align*}
\begin{pmatrix}
\mid & & \mid \\
\Delta\mat{f}^{\mathcal{Z}_1} & \cdots & \Delta\mat{f}^{\mathcal{Z}_v}\\
\mid & & \mid
\end{pmatrix}
= \mat{M}\mat{N}
\end{align*}
Note that $\mathcal{Z}_1, \dots, \mathcal{Z}_v$ do not need to be ordered.

A linear combination of $\Delta\mat{f}^{\mathcal{Z}_1}, \dots, \Delta\mat{f}^{\mathcal{Z}_v}$ with scale factors $\alpha_1, \dots, \alpha_v$ is given by
\begin{align}
\Delta\mat{f}(\mat{\alpha}) = \mat{M}\mat{N}\mat{\alpha}.\label{eq:alphalincombination}
\end{align}
Condition (\ref{eq:diffconditionmat}) then becomes:
\begin{align}
\mat{N}^*\mat{M}\mat{N}\mat{\alpha} &= -\mat{N}^*\mat{f}\\
\intertext{with pseudo-solution}
\mat{\alpha} &= -\left(\mat{N}^*\mat{M}\mat{N}\right)^+\mat{N}^*\mat{f}.\label{eq:alphasolution}
\end{align}
Finally, combining (\ref{eq:alphalincombination}) and (\ref{eq:alphasolution}) gives:
\begin{align*}
\Delta\mat{f} = -\mat{M}\mat{N} \left(\mat{N}^*\mat{M}\mat{N}\right)^+\mat{N}^*\mat{f},
\end{align*}
in agreement with the result.\hfill$\neg$\leafNE
\end{intuition}

A rigorous proof of Theorem \ref{thm:lineflowsafterfailures} is given in \cite{Ronellenfitsch2017}. This only needs to be adapted to our notation.

In (\ref{eq:alphasolution}) we take the \emph{pseudo-inverse}, instead of the general inverse, to account for a set of failures that disconnects the network. We discuss the implications of this modification in Section \ref{sec:discussionpowerislands}.
\end{document}
